Nick Vosseteig

2014-10-24

Building, Wiring, Programming

\begin{tabular}{|p{5cm}|p{5cm}|}
 \hline
 Building&
This week I built a frame for the robot and added the wheels, using holonomic wheels in the front of the robot to allow it to turn more easily. Matt and David worked on building a prototype for the intake device, but did not complete it. 
 \\
 \hline
Wiring&
I also did all the necessary wiring for the frame so now the 4 DC motors necessary to drive are wired.
 \\
 \hline
Programming& I wrote a simple program that allows us to control/test the robot with the Logitech controller.
 \\
 \hline
\end{tabular}

\section*{Programming}
This program allows us to drive the robot frame.
\begin{lstlisting}[style = RobotC]	
#include "JoystickDriver.c"

task main()
{
	while(true)
	{
		getJoystickSettings(joystick);

		if(joystick.joy1_y1<20 && joystick.joy1_y1>-20){
			motor[left] = 0;
			motor[left2] = 0;
			}else{
			motor[left] = -joystick.joy1_y1/2;
			motor[left2] = joystick.joy1_y1/2;
		}
		if(joystick.joy1_y2<20 && joystick.joy1_y2>-20){
			motor[right] = 0;
			motor[right2] = 0;
			}else{
			motor[right] = joystick.joy1_y2/2;
			motor[right2] = joystick.joy1_y2/2;
		}
	}
}
\end{lstlisting}
We did not get a chance to test the code out yet, so we will have to do it next week.

\section*{Building and wiring}
I built and wired the frame of the robot which has 4 DC motors, one for each wheel. The front wheels are 4 holonomic wheels set 2 on each side for stability and better ability to turn. The building and wiring went pretty well since I had done it before and knew what to do. The wires are still a mess, and I need to clean them up in the future.
